\chapter{Research Design and Methodology}

\section{Manuscript Format Requirements}

Except as specifically superseded by directions from the candidate's major department, the general rules with respect to manuscript format shall follow those provided below. Some of the guidance is adapted from K. L. Turabian, \textit{A Manual for Writers of Term Papers, Theses, and Dissertations}, 8th Edition, University of Chicago Press, Chicago, 2013.

The preparation of the thesis or dissertation and all required copies is the responsibility of the student, unless departmental policies specify otherwise.

\subsection{Title}

The title is the first element read by the audience. It should clearly announce the topic and communicate the conceptual framework of the thesis or dissertation. Titles should include relevant keywords that are informative to both readers and search or indexing systems.

\subsection{Font}

A single, readable, and widely available typeface must be used throughout the document, such as Times New Roman, Arial, or Helvetica. If a less common typeface is used, the font must be embedded in the electronic file. Ornamental or decorative typefaces should be avoided. In general, the body text should be set in at least 10-point or 12-point font.

\subsection{Title Page}

The first page of the PDF must be the title page. The title page must follow the format specified in the official template. No additional information may be included beyond what is defined in the template.

\subsection{Copyright Page}

If the student chooses to assert copyright, the second page of the PDF must be the copyright page, formatted according to the template. If the student does not wish to assert copyright, this choice must be clearly indicated on the Submission Checklist.

\subsection{Acknowledgments Page}

All theses and dissertations must include an Acknowledgments section. This section is used to recognize mentors, colleagues, individuals, or institutions that supported the research or provided special assistance, such as consultation or technical aid. Any owners of copyrighted materials who granted permission for reproduction of their work must be acknowledged. All sources of financial support, including external grants, fellowships, awards, or self-supported funding, must be disclosed, and identifying grant numbers should be included where applicable.

Acknowledgment of funding sources is an ethical requirement in research publications, including theses and dissertations, as it provides proper recognition to sponsors and publicly discloses financial support.

For doctoral submissions, the doctoral committee must also be listed in the Acknowledgments section, and the chair of the committee must be identified. The doctoral committee must not be listed on the title page.

\subsection{Abstract}

The abstract will be made publicly available through the institutional repository. Footnotes, references, and unexplained abbreviations must not be included. There is no strict word limit; however, the abstract should be concise and clearly summarize the work.

\subsection{Table of Contents}

The table of contents must include page numbers for all listed sections.

\subsection{List of Tables}

All tables must be listed with their titles and corresponding page numbers.

\subsection{List of Figures and Illustrations}

All figures and illustrations must be listed with their titles and corresponding page numbers.

\subsection{Body}

The body of the thesis or dissertation should be organized into the following sections:

\begin{itemize}
    \item Introduction
    \item Main body, with major divisions and important minor divisions clearly indicated using consistent headings
    \item Summary and conclusions, highlighting key findings and conclusions. For engineering and science theses and dissertations, this section often also includes recommendations for future research
    \item References
    \item Appendices, with each appendix titled and listed in the Table of Contents
\end{itemize}

\subsection{Pagination}

Each page of the thesis or dissertation must be assigned a page number. The following pagination scheme is generally accepted:

\begin{itemize}
    \item The Title Page and Copyright Page are not numbered, although they are included in the page count.
    \item Preliminary pages use lowercase Roman numerals (i, ii, iii, iv, etc.). Page numbering begins with iii; the title page counts as page i and the copyright page as page ii, but the numbers do not appear.
    \item The remainder of the document, including the main text, figures, tables, appendices, and references, uses Arabic numerals (1, 2, 3, \ldots). Page numbering begins with 1 and continues consecutively to the end of the document. The use of letter suffixes such as 10a or 10b should be avoided.
\end{itemize}

\subsection{Footnotes}

If footnotes are used, they must appear at the bottom of the page, separated from the main text by a 1.5-inch horizontal line starting at the left margin. The first line of each footnote should be indented 0.5 inches and identified by a superscript numeral corresponding to the reference in the text. Footnotes must be numbered consecutively within each chapter.

\subsection{Reproduction of Data}

The data on which the thesis or dissertation is based must be made accessible to the reader in substantially complete form. In general, raw data should be reproduced in one or more appendices and made available through the institutional repository, a stable departmental or advisor-supported website, or an appropriate external repository relevant to the field.

For extensive data derived from readily available published sources, detailed citations may be sufficient, provided the data are included in an appendix when possible or a stable URL is supplied. Any deviation from full disclosure must be explicitly approved by the M.S. thesis advisor(s) or Ph.D. dissertation committee and fully explained in the document.

\subsection{Reproduction of Materials}

All instruments, analytical procedures, apparatus, and other critical components used in the study must be described in sufficient detail to allow replication. Apparatus should normally be documented using engineering drawings and photographs. Instruments should be reproduced in full using figures or drawings unless they are readily available from other sources.

Analytical procedures must be fully specified either through detailed discussion or by citation, with supplementary explanations provided in appendices as needed. Computer calculations essential to the central arguments of the research must be clearly and fully explained. If computer programs are developed by the student, a program listing and minimal documentation must be included in the thesis or dissertation.

Program listings and documentation are normally placed in an appendix. If the computer work is extensive and deemed too lengthy for inclusion, a summarized presentation using tables may be provided, accompanied by a separate internal report. Standard subroutines or commercially available software packages are exempt from this requirement but must be clearly cited, and their sources identified.

\subsection{References}

Citations must be consistent and standardized throughout the thesis or dissertation. The citation style should conform to that used in a standard professional journal within the candidate’s field. The Harvard citation style is commonly used in engineering and science disciplines.

The following journals are recommended as style references:

\begin{itemize}
    \item Biomedical Engineering: \textit{Annals of Biomedical Engineering}, \textit{Journal of Biomechanical Engineering}
    \item Chemical Engineering: \textit{Langmuir}, \textit{Optimization and Engineering}
    \item Civil and Environmental Engineering: ASCE journals (e.g., \textit{Journal of Environmental Engineering}, \textit{Journal of Transportation Engineering})
    \item Electrical and Computer Engineering: \textit{Proceedings of the IEEE}
    \item Engineering and Public Policy: \textit{Science}, \textit{Proceedings of the IEEE}
    \item Mechanical Engineering: \textit{Transactions of the American Society of Mechanical Engineers}
    \item Materials Science and Engineering: \textit{Metallurgical Transactions}
\end{itemize}

\subsection{Additional Guidance}

Students should refer to the institutional thesis and dissertation submission guidelines for detailed requirements regarding margins, paper size, line spacing, and additional formatting specifications not covered above.

