\chapter{First Appendix}
The appendix provides supplementary material that supports the main content of the thesis but is not essential to the continuity of the primary narrative. It is intended to present detailed information that would otherwise disrupt the flow of the main text, while remaining available for reference by readers seeking additional technical depth or clarification.

Materials included in the appendix may consist of extended mathematical derivations, detailed algorithms, additional experimental results, raw or preprocessed datasets, implementation details, configuration settings, source code excerpts, survey instruments, questionnaires, or supporting figures and tables. Content placed in the appendix should be clearly referenced from the main body of the thesis where relevant.

All appendices must be listed in the Table of Contents. Figures, tables, equations, and references appearing in the appendix should be numbered consistently, either by continuing the numbering from the main text or by using appendix-specific numbering, in accordance with the chosen formatting style.

The appendix is considered an integral part of the thesis and must adhere to the same standards of clarity, formatting, and scholarly rigor as the main chapters. Any data, software, or materials necessary to reproduce the results presented in the thesis should be included in the appendix or made accessible through a stable external repository, with appropriate references provided.

While appendices may contain extensive or highly technical material, authors should ensure that all content is well organized, clearly explained, and directly relevant to the research. Information that is not referenced or does not meaningfully support the thesis should not be included.
