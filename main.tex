% !TeX program = pdflatex
% !TeX bib = bibtex
\documentclass{cmklthesis}

% Additional packages if needed
\usepackage{lipsum} % For placeholder text
\usepackage[colorlinks=true, linkcolor=black, citecolor=blue, urlcolor=blue]{hyperref}


% Thesis Information
\title{Title of your thesis }
\author{Firstname Middleinitial Lastname}
\degree{Doctor of Philosophy}
\department{Artificial Intelligence and Computer Engineering}
\degreedate{May}{2026}
\previousdegrees{B.S., Engineering Department, University degree was obtained\\
M.S., Engineering Department, University degree was obtained}


\begin{document}

% Title Page
\maketitle

% Copyright Page
% Comment out if you do not wish to assert copyright
\makecopyright

% Dedication Page
% Comment out if you do not wish to include the dedication page
\begin{dedication}
    To my family...
\end{dedication}

% Acknowledgements
\begin{acknowledgments}
    This section is used to thank mentors and colleagues or name the individuals or institutions that supported your research or provided special assistance, such as consultation or aid. Acknowledge any owners of copyrighted materials that have granted you permission to reproduce their work. Describe all sources of funding from outside grants, fellowships, awards, or self-supported funding. For any grants, include the identifying number. Acknowledgment of the source(s) of support is important ethically in all research publications and presentations, including theses, to give the sponsors the recognition they deserve, and also to disclose publicly the organization or persons funding the research.

    For doctoral submissions, the doctoral committee must also be listed in the Acknowledgments, and the chair of the committee must be identified. The doctoral committee should not be listed on the title page.
\end{acknowledgments}

% Abstract
\begin{abstract}
    This section is for the abstract. Keep in mind that the abstract will be published in the ProQuest Dissertations and Theses database (PQDT) with no further editing or revisions. Do not include footnotes, references, or unexplained abbreviations. There is no word limit on the abstract, however it should be concise.
\end{abstract}

% List of Abbreviations
\begin{abbreviations}
    \begin{tabular}{ll}
        AI & Artificial Intelligence \\
        ANN & Artificial Neural Networks \\
        CNN & Convolutional Neural Network \\
        FPR & False Positive Rate \\
    \end{tabular}
\end{abbreviations}

% Table of Contents
\tableofcontents
\clearpage

% List of Tables
\listoftables
\clearpage

% List of Figures
\listoffigures
\clearpage

% Start Main Body (Arabic numbering)
\startmainbody

\chapter{Introduction}
\lipsum[1-2]
\cite{einstein}

\section{Subheading}
\lipsum[3]
This is a citation \cite{einstein}.

\chapter{Research Design and Methodology}
\lipsum[4-5]
This is the second citation \cite{bishop2006}
This is the third citation \cite{goodfellow2016}
This is the fourth citation \cite{han2021transformer}

\section{Subheading}
\lipsum[6]

\begin{figure}[tb]
    \centering
    \includegraphics[width=0.7\textwidth]{figures/image-01-million_billion_trillion.png}
    \caption{The trend of actual versus perceived size of a number. \cite{einstein}. We are also
    going to add a relatively long caption to check if the List of Figures works properly at the beginning of the document.}
    \label{fig:xkcd1}
\end{figure}

\begin{figure}[tb]
    \centering
    \includegraphics[width=0.7\textwidth]{figures/image-02-phd092809s.png}
    \caption{"Vacation Relaxation?" by Jorge Cham
www.phdcomics.com. A nice comic form PHD Comic's. Learn more about the figure input notation in the Figure chapter, or in the hand book.}
    \label{fig:phdcomics1}
\end{figure}


\subsection{Table}
\lipsum[1-3]
\begin{table}[tb]
	\caption{This is where we have fun testing encoding}
	\begin{center}
		\begin{tabular}{|c|c|c|}
			\hline
			& Normal & Math \\
			\hline
			The greater than: & > & $>$ \\
			\hline
			The less than: & < & $<$ \\
			\hline
			The tilde: & \textasciitilde{} & $\sim$ \\
			\hline
		\end{tabular}
	\end{center}
    \label{tab:encoding}
\end{table}

\begin{table}
    \centering
    \caption{Table that should take up more than 50\% of the page to stand on its own. Note that it is automatically placed on it's own page when leaving off the placement (eg. [ht] is removed after \textbackslash begin\{table\})}
    \label{tab:my_labelTable}
\begin{tabular}{lcrcl}
\hline
Long Table & Big Table & 50\% of the page & Other & And another \\ \hline
1          & 2         & 3                & 4          & 5           \\
A          & B         & C                & D          & E           \\
6          & 7         & 8                & 9          & 10          \\
a          & b         & c                & d          & e          \\
11         & 21        & 31               & 41         & 51          \\
\textbf{A} & \textbf{B}& \textbf{C}       & \textbf{D} & \textbf{E}  \\
61         & 71        & 81               & 91         & 101         \\
\textbf{a} & \textbf{b}& \textbf{c}       & \textbf{d} & \textbf{e} \\
111        & 211       & 311              & 411        & 511         \\
\textit{A} & \textit{B}& \textit{C}       & \textit{D} & \textit{E}  \\
611        & 711       & 811              & 911        & 1011        \\
\textit{a} & \textit{b}& \textit{c}       & \textit{d} & \textit{e}  \\
1          & 2         & 3                & 4          & 5           \\
A          & B         & C                & D          & E           \\
6          & 7         & 8                & 9          & 10          \\
a          & b         & c                & d          & e          \\
11         & 21        & 31               & 41         & 51          \\
\textbf{A} & \textbf{B}& \textbf{C}       & \textbf{D} & \textbf{E}  \\
61         & 71        & 81               & 91         & 101         \\
\textbf{a} & \textbf{b}& \textbf{c}       & \textbf{d} & \textbf{e} \\
111        & 211       & 311              & 411        & 511         \\
\textit{A} & \textit{B}& \textit{C}       & \textit{D} & \textit{E}  \\
611        & 711       & 811              & 911        & 1011        \\
\textit{a} & \textit{b}& \textit{c}       & \textit{d} & \textit{e}  \\ \hline
\end{tabular}
\end{table}

\chapter{Analysis}
\lipsum[7-8]
This is cross reference to Figure~\ref{fig:xkcd1} and Figure~\ref{fig:phdcomics1} and Table~\ref{tab:encoding}.
\lipsum[9]\footnote{This is the footnote text. This is the footnote text. This is the footnote text. This is the footnote text. This is the footnote text. This is the footnote text. This is the footnote text. }


\chapter{Summary and Conclusions}
\lipsum[9]\footnote{If footnotes are needed, they should be placed at the bottom of the page below a 1.5-inch underscore (starting at the left border). }
\lipsum[9]\footnote{The first line of each footnote should be indented 0.5 inches and identified by a raised numeral corresponding to that used in the test. Footnotes should be numbered consecutively throughout each chapter. }

% Bibliography
\bibliographystyle{IEEEtran}
\bibliography{references}

% Appendices
\appendix
\chapter{First Appendix}
\lipsum[10]


\end{document}
