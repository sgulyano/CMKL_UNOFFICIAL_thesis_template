% !TeX program = pdflatex
% !TeX bib = bibtex
\documentclass{cmklthesis}

% Additional packages if needed
\usepackage[colorlinks=true, linkcolor=black, citecolor=blue, urlcolor=blue]{hyperref}


% Thesis Information
\title{Title of your thesis }
\author{Firstname Middleinitial Lastname}
\degree{Doctor of Philosophy}
\department{Artificial Intelligence and Computer Engineering}
\degreedate{May}{2026}
\previousdegrees{B.S., Engineering Department, University degree was obtained\\
M.S., Engineering Department, University degree was obtained}


\begin{document}

% Title Page
\maketitle

% Copyright Page
% Comment out if you do not wish to assert copyright
\makecopyright

% Dedication Page
% Comment out if you do not wish to include the dedication page
\begin{dedication}
    To my family...
\end{dedication}

% Acknowledgements
\begin{acknowledgments}
    This section is used to thank mentors and colleagues or name the individuals or institutions that supported your research or provided special assistance, such as consultation or aid. Acknowledge any owners of copyrighted materials that have granted you permission to reproduce their work. Describe all sources of funding from outside grants, fellowships, awards, or self-supported funding. For any grants, include the identifying number. Acknowledgment of the source(s) of support is important ethically in all research publications and presentations, including theses, to give the sponsors the recognition they deserve, and also to disclose publicly the organization or persons funding the research.

    For doctoral submissions, the doctoral committee must also be listed in the Acknowledgments, and the chair of the committee must be identified. The doctoral committee should not be listed on the title page.

    If the acknowledgments are more than one page, do not repeat the title. Leave two blank lines between the title and the first line of the text. Double-space the test of the acknowledgments, and format it to match the main text.

    (Note: begin roman numeral pagination on this page as page number iii (ii if copyright is not asserted); continue until the body of the thesis begins.)

\end{acknowledgments}

% Abstract
\begin{abstract}
    This section is reserved for the abstract. Please note that the abstract will be published in the institutional repository without further editing or revision. Do not include footnotes, references, or unexplained abbreviations. While there is no strict word limit, the abstract should be clear and concise.
\end{abstract}

% List of Abbreviations
\begin{abbreviations}
    \begin{tabular}{ll}
        AI & Artificial Intelligence \\
        ANN & Artificial Neural Networks \\
        CNN & Convolutional Neural Network \\
        FPR & False Positive Rate \\
    \end{tabular}
\end{abbreviations}

% Table of Contents
\tableofcontents
\clearpage

% List of Tables
\listoftables
\clearpage

% List of Figures
\listoffigures
\clearpage

% Start Main Body (Arabic numbering)
\startmainbody

\chapter{Introduction}
The thesis or dissertation must be a document of the best professional standards. It is also good practice for the student to prepare a document that meets the criteria for publication in the relevant professional journals. As the original copy of the thesis or dissertation will be kept in the University Libraries, and copied for microfilming and other purposes, the paper and the production must conform to standards of long archive life and clear reproducibility. In addition, an electronic copy of the thesis or dissertation is required by the College of Engineering for archiving by the department. These instructions provide a guide for the production of a high-quality thesis or dissertation document, and formatting specifications to ensure some basic consistency among engineering theses and dissertations. It is the responsibility of the student to see that these guidelines are met, and the responsibility of the department to confirm this before submitting the thesis or dissertation to the College of Engineering Dean for approval. While preparing your thesis or dissertation the guidelines in the booklet, Publishing Your Doctoral Dissertation with UMI Dissertation Publishing, provided by ProQuest/UMI should be followed. Download a copy of this booklet, or obtain it from the department's graduate coordinator. Other requirements specific to the college are provided below. \cite{einstein}


\section{Submission Procedure}

The College of Engineering requires that all theses and dissertations be submitted to both the Carnegie Mellon University Institutional Repository and the ProQuest ETD Administrator Repository. This submission is completed through the ProQuest ETD Administrator website.

\subsection{ProQuest}

ProQuest offers two publishing options: \textit{Traditional Publishing} and \textit{Open Access Publishing PLUS}. Under all publishing options, students retain the copyright to their work. For an additional fee, ProQuest can officially register the student’s copyright with the U.S. Copyright Office. While registration is not required to maintain copyright ownership, it may provide certain legal benefits. For more information, students should consult the UMI Copyright Guide.

\subsubsection{Traditional Publishing}

Under the Traditional Publishing option, students enter into an agreement granting ProQuest a non-exclusive license to publish the abstract and to duplicate and distribute the dissertation. The abstract, bibliography, and other metadata of the thesis or dissertation will be included in the \textit{ProQuest Dissertations and Theses} (PQDT) database. ProQuest pays authors a 10\% royalty on any sales of their work.

\subsubsection{Open Access Publishing through ProQuest PLUS}

Under the Open Access Publishing PLUS option, students enter into an agreement granting ProQuest a non-exclusive license to publish their work in the \textit{ProQuest Dissertations and Theses Open Database} and make it available for free download. Students do not receive royalties under this option. A one-time upfront fee is required. Additional information on Open Access Publishing PLUS is available on the ProQuest website.

\subsection{Carnegie Mellon University Institutional Repository}

During the ProQuest ETD Administrator submission process, students are required to publish their thesis or dissertation in the Carnegie Mellon University Institutional Repository. This repository, supported by the University Libraries, provides online open access to scholarly work produced by Carnegie Mellon University faculty and students. Students may choose to restrict access to campus-only (archival) access. There is no fee for submission.

Publishing a thesis or dissertation in the Institutional Repository does not affect authorial copyright ownership. All approved embargo options will be honored.

\subsection{Embargo Options}

An embargo allows the release of a thesis or dissertation to be delayed for a limited period of time. Embargos are commonly requested due to pending patents, the inclusion of material that cannot be released because of copyright restrictions, or the intention to publish all or part of the work in a journal or book.

\subsection{Supplementary Materials}

Supplementary materials, such as raw data underlying the research, should be uploaded during the ProQuest ETD Administrator submission process. These materials will be made available online with the thesis or dissertation in the Institutional Repository, or provided on a CD or DVD if a printed copy is requested.

\subsection{Departmental Copies}

The thesis or dissertation may also be archived by the department on a non-public server. In some departments, authors may have the option to post the thesis or dissertation on a publicly accessible website maintained by the department. Students should consult their departmental handbook for specific policies and requirements.


\section{Required Documentation}

The following documents must be submitted to the College of Engineering Graduate School in addition to the electronic submission of the thesis or dissertation:

\begin{itemize}
    \item A PDF of the completed thesis or dissertation
    \item A Signature Page, signed by the advisor(s) and Department Head(s)
    \item A Committee Page, signed by all committee members
    \item A Submission Checklist confirming proper document formatting and the copyright assertion decision
\end{itemize}

\subsection{Signature Page}

Separate from the PDF of the thesis or dissertation, a Signature Page is required and must follow the format specified for regular degrees and joint degrees. The original Signature Page must be signed by the thesis or dissertation advisor(s), the Department Head, and the Dean or Associate Dean for Graduate and Faculty Affairs of the College of Engineering. Only one original Signature Page should accompany the thesis or dissertation submitted to the Department Head and Dean for review and approval.

\subsection{Committee Page}

Separate from the PDF of the thesis or dissertation, a Committee Page is required. The original Committee Page must be signed by all committee members for doctoral works and by all readers for master’s works.

\subsection{Submission Checklist}

Separate from the PDF of the thesis or dissertation, a Submission Checklist is required. The checklist must be carefully reviewed to ensure that all submission requirements have been satisfied. The Submission Checklist must be completed and signed by the student.

For additional details, refer to the College of Engineering thesis and dissertation policies available at
\url{https://engineering.cmu.edu/education/academic-policies/graduate-policies/thesis-dissertation.html}.


\chapter{Research Design and Methodology}

\section{Manuscript Format Requirements}

Except as specifically superseded by directions from the candidate's major department, the general rules with respect to manuscript format shall follow those provided below. Some of the guidance is adapted from K. L. Turabian, \textit{A Manual for Writers of Term Papers, Theses, and Dissertations}, 8th Edition, University of Chicago Press, Chicago, 2013.

The preparation of the thesis or dissertation and all required copies is the responsibility of the student, unless departmental policies specify otherwise.

\subsection{Title}

The title is the first element read by the audience. It should clearly announce the topic and communicate the conceptual framework of the thesis or dissertation. Titles should include relevant keywords that are informative to both readers and search or indexing systems.

\subsection{Font}

A single, readable, and widely available typeface must be used throughout the document, such as Times New Roman, Arial, or Helvetica. If a less common typeface is used, the font must be embedded in the electronic file. Ornamental or decorative typefaces should be avoided. In general, the body text should be set in at least 10-point or 12-point font.

\subsection{Title Page}

The first page of the PDF must be the title page. The title page must follow the format specified in the official template. No additional information may be included beyond what is defined in the template.

\subsection{Copyright Page}

If the student chooses to assert copyright, the second page of the PDF must be the copyright page, formatted according to the template. If the student does not wish to assert copyright, this choice must be clearly indicated on the Submission Checklist.

\subsection{Acknowledgments Page}

All theses and dissertations must include an Acknowledgments section. This section is used to recognize mentors, colleagues, individuals, or institutions that supported the research or provided special assistance, such as consultation or technical aid. Any owners of copyrighted materials who granted permission for reproduction of their work must be acknowledged. All sources of financial support, including external grants, fellowships, awards, or self-supported funding, must be disclosed, and identifying grant numbers should be included where applicable.

Acknowledgment of funding sources is an ethical requirement in research publications, including theses and dissertations, as it provides proper recognition to sponsors and publicly discloses financial support.

For doctoral submissions, the doctoral committee must also be listed in the Acknowledgments section, and the chair of the committee must be identified. The doctoral committee must not be listed on the title page.

\subsection{Abstract}

The abstract will be made publicly available through the institutional repository. Footnotes, references, and unexplained abbreviations must not be included. There is no strict word limit; however, the abstract should be concise and clearly summarize the work.

\subsection{Table of Contents}

The table of contents must include page numbers for all listed sections.

\subsection{List of Tables}

All tables must be listed with their titles and corresponding page numbers.

\subsection{List of Figures and Illustrations}

All figures and illustrations must be listed with their titles and corresponding page numbers.

\subsection{Body}

The body of the thesis or dissertation should be organized into the following sections:

\begin{itemize}
    \item Introduction
    \item Main body, with major divisions and important minor divisions clearly indicated using consistent headings
    \item Summary and conclusions, highlighting key findings and conclusions. For engineering and science theses and dissertations, this section often also includes recommendations for future research
    \item References
    \item Appendices, with each appendix titled and listed in the Table of Contents
\end{itemize}

\subsection{Pagination}

Each page of the thesis or dissertation must be assigned a page number. The following pagination scheme is generally accepted:

\begin{itemize}
    \item The Title Page and Copyright Page are not numbered, although they are included in the page count.
    \item Preliminary pages use lowercase Roman numerals (i, ii, iii, iv, etc.). Page numbering begins with iii; the title page counts as page i and the copyright page as page ii, but the numbers do not appear.
    \item The remainder of the document, including the main text, figures, tables, appendices, and references, uses Arabic numerals (1, 2, 3, \ldots). Page numbering begins with 1 and continues consecutively to the end of the document. The use of letter suffixes such as 10a or 10b should be avoided.
\end{itemize}

\subsection{Footnotes}

If footnotes are used, they must appear at the bottom of the page, separated from the main text by a 1.5-inch horizontal line starting at the left margin. The first line of each footnote should be indented 0.5 inches and identified by a superscript numeral corresponding to the reference in the text. Footnotes must be numbered consecutively within each chapter.

\subsection{Reproduction of Data}

The data on which the thesis or dissertation is based must be made accessible to the reader in substantially complete form. In general, raw data should be reproduced in one or more appendices and made available through the institutional repository, a stable departmental or advisor-supported website, or an appropriate external repository relevant to the field.

For extensive data derived from readily available published sources, detailed citations may be sufficient, provided the data are included in an appendix when possible or a stable URL is supplied. Any deviation from full disclosure must be explicitly approved by the M.S. thesis advisor(s) or Ph.D. dissertation committee and fully explained in the document.

\subsection{Reproduction of Materials}

All instruments, analytical procedures, apparatus, and other critical components used in the study must be described in sufficient detail to allow replication. Apparatus should normally be documented using engineering drawings and photographs. Instruments should be reproduced in full using figures or drawings unless they are readily available from other sources.

Analytical procedures must be fully specified either through detailed discussion or by citation, with supplementary explanations provided in appendices as needed. Computer calculations essential to the central arguments of the research must be clearly and fully explained. If computer programs are developed by the student, a program listing and minimal documentation must be included in the thesis or dissertation.

Program listings and documentation are normally placed in an appendix. If the computer work is extensive and deemed too lengthy for inclusion, a summarized presentation using tables may be provided, accompanied by a separate internal report. Standard subroutines or commercially available software packages are exempt from this requirement but must be clearly cited, and their sources identified.

\subsection{References}

Citations must be consistent and standardized throughout the thesis or dissertation. The citation style should conform to that used in a standard professional journal within the candidate’s field. The Harvard citation style is commonly used in engineering and science disciplines.

The following journals are recommended as style references:

\begin{itemize}
    \item Biomedical Engineering: \textit{Annals of Biomedical Engineering}, \textit{Journal of Biomechanical Engineering}
    \item Chemical Engineering: \textit{Langmuir}, \textit{Optimization and Engineering}
    \item Civil and Environmental Engineering: ASCE journals (e.g., \textit{Journal of Environmental Engineering}, \textit{Journal of Transportation Engineering})
    \item Electrical and Computer Engineering: \textit{Proceedings of the IEEE}
    \item Engineering and Public Policy: \textit{Science}, \textit{Proceedings of the IEEE}
    \item Mechanical Engineering: \textit{Transactions of the American Society of Mechanical Engineers}
    \item Materials Science and Engineering: \textit{Metallurgical Transactions}
\end{itemize}

\subsection{Additional Guidance}

Students should refer to the institutional thesis and dissertation submission guidelines for detailed requirements regarding margins, paper size, line spacing, and additional formatting specifications not covered above.



\chapter{Analysis}

\section{Using Figures, Tables, Citations, and Cross-References}

This section demonstrates how to include citations, figures, tables, cross-references, and footnotes in a thesis or dissertation prepared using \LaTeX.

\subsection{Citations}

This is the second citation \cite{bishop2006}.  
This is the third citation \cite{goodfellow2016}.  
This is the fourth citation \cite{han2021transformer}.

\subsection{Figures}
Figures are used to illustrate concepts, present visual results, or support explanations in the text. Each figure must include a descriptive caption and be referenced in the main text.

\begin{figure}[tb]
    \centering
    \includegraphics[width=0.7\textwidth]{figures/image-01-million_billion_trillion.png}
    \caption{The trend of actual versus perceived size of a number \cite{einstein}. This caption is intentionally long to demonstrate proper formatting and correct appearance in the List of Figures.}
    \label{fig:xkcd1}
\end{figure}

\begin{figure}[tb]
    \centering
    \includegraphics[width=0.7\textwidth]{figures/image-02-phd092809s.png}
    \caption{``Vacation Relaxation?'' by Jorge Cham, \texttt{www.phdcomics.com}. This example illustrates how to include external images and descriptive captions in a thesis or dissertation.}
    \label{fig:phdcomics1}
\end{figure}

\subsection{Tables}

Tables are used to present structured data in rows and columns. Each table must include a caption and should be referenced in the text.


\begin{table}[tb]
	\centering
	\caption{Examples of character encoding in text and math modes}
	\label{tab:encoding}
	\begin{tabular}{|c|c|c|}
		\hline
		 & Normal Text & Math Mode \\ \hline
		Greater than & > & $>$ \\ \hline
		Less than & < & $<$ \\ \hline
		Tilde & \textasciitilde{} & $\sim$ \\ \hline
	\end{tabular}
\end{table}

\begin{table}
    \centering
    \caption{An example of a large table that occupies more than half of a page. When no placement option is specified, \LaTeX{} may place the table on a separate page automatically.}
    \label{tab:my_labelTable}
\begin{tabular}{lcrcl}
\hline
Long Table & Big Table & 50\% of the page & Other & And another \\ \hline
1          & 2         & 3                & 4          & 5           \\
A          & B         & C                & D          & E           \\
6          & 7         & 8                & 9          & 10          \\
a          & b         & c                & d          & e           \\
11         & 21        & 31               & 41         & 51          \\
\textbf{A} & \textbf{B}& \textbf{C}       & \textbf{D} & \textbf{E}  \\
61         & 71        & 81               & 91         & 101         \\
\textbf{a} & \textbf{b}& \textbf{c}       & \textbf{d} & \textbf{e} \\
111        & 211       & 311              & 411        & 511         \\
\textit{A} & \textit{B}& \textit{C}       & \textit{D} & \textit{E}  \\
611        & 711       & 811              & 911        & 1011        \\
\textit{a} & \textit{b}& \textit{c}       & \textit{d} & \textit{e}  \\ \hline
\end{tabular}
\end{table}

\subsection{Footnote}

This chapter demonstrates how to reference figures and tables within the text. For example, Figure~\ref{fig:xkcd1} and Figure~\ref{fig:phdcomics1} illustrate proper figure usage, while Table~\ref{tab:encoding} provides an example of table referencing.

Footnotes may be used to provide additional clarification without interrupting the flow of the main text.\footnote{This is an example of a footnote. Footnotes should be used sparingly and placed at the bottom of the page.} This is the second footnote.\footnote{If footnotes are needed, they should be placed at the bottom of the page below a 1.5-inch underscore (starting at the left border). } This is the last footnote. \footnote{The first line of each footnote should be indented 0.5 inches and identified by a raised numeral corresponding to that used in the test. Footnotes should be numbered consecutively throughout each chapter. }



\chapter{Summary and Conclusions}
This chapter provides a comprehensive summary of the work presented in this thesis. It synthesizes the main findings and key contributions of the research, discusses their significance in the context of the research objectives, and presents the overall conclusions drawn from the study. Where appropriate, the chapter also outlines limitations of the work and highlights potential directions for future research.


% Bibliography
\bibliographystyle{IEEEtran}
\bibliography{references}

% Appendices
\appendix

\appendix{Magical Encoding Awesomeness}\label{app:encoding}
\ref{tab:encoding} shows how several symbols appear in the rendered document.

\begin{table}[H]
	\caption{\label{tab:encoding}This is where we have fun testing encoding}
	\begin{center}
		\begin{tabular}{|c|c|c|}
			\hline
			& Normal & Math \\
			\hline
			The greater than: & > & $>$ \\
			\hline
			The less than: & < & $<$ \\
			\hline
			The tilde: & \textasciitilde{} & $\sim$ \\
			\hline
		\end{tabular}
	\end{center}
\end{table}

\subsection{Test Appendix Sub-Section}\label{sec:longtable}
\ref{tab:longtable} is an example of a very large ``longtable.''
\begin{landscape}
\begin{longtable}{|>{\centering}p{1.02in}|>{\centering}p{1.15in}|>{\centering}p{1in}|>{\centering}p{0.7in}|>{\centering}p{0.7in}|>{\centering}p{0.67in}|>{\centering}p{2.55in}|} %
	\endfirsthead % Remove this line to use the main header for the first page
	\hline%
	Age & Formation  & Thickness (feet)   & Thickness (feet)  & Thickness (feet)  & Aquifer?  & Lithology 	
	\endhead%
	\caption{Stratigraphy of the Granite Mountains and Lost Creek areas\label{tab:longtable}}\\ %
	\hline
	Age & Formation \footnote{Only major unconformities shown, indicated by break in table.} & Thickness (feet) \footnote{Generalized thicknesses from.}  & Thickness (feet) \footnote{Thicknesses shown are approximate and apply to Lost Creek vicinity
	only.} & Thickness (feet) \footnote{Thicknesses shown are from a public screened dataset of logged formation
	tops from the 12 townships surrounding Lost Creek. } & Aquifer? \footnote{Aquifer designations \textendash{} Lost Creek vicinity only.%
	} & Lithology \tabularnewline
	\hline 
	Quaternary  & Alluvium & - & 0-20 & - & Yes & Sands and clays derived chiefly from the Tertiary formations in the
	area. \tabularnewline
	\hline 
	Paleocene & Fort Union  & up to 3,000 & 4,650 & 6,500? & Yes & Consists of alternating fine to coarse grained sandstone siltstone
	and mudstone. Contains various layers of lignitic coal beds. \tabularnewline
	\hline
	\hline 
	Cretaceous  & Lance  & 1,700 to 2,700 & 2,950 & 4,000? & Yes & Interbedded sandstone, siltstone and mudstone. Gray to brownish gray.
	Locally carbonaceous. Sandstone is white to grayish orange. \tabularnewline
	\hline 
	Cretaceous & Fox Hills  &  & 550 & 1,800? & No & Consists of coarsening upward shale and fine-grained sand with thin
	coal beds near the top. Represents a transition from marine to non-marine
	environment. Grades into Lewis Shale at the base. \tabularnewline
	\hline 
	Cretaceous & Lewis Shale  & 1,250 & 1,200 & 1,050 to 2,000 & No & Interbedded dark-gray and olive-gray shale and olive-gray sandstone. \tabularnewline
	\hline
	\hline 
	Cretaceous & Mesaverde Group  & 0 to 1,000 & 800 & 300 to 500? & No & Gray to dark gray shales with interbedded buff to tan fine to medium
	grained sandstones. \tabularnewline
	\hline 
	Cretaceous & Steele and Niobrara Shales  & Cody Shale 4,500 to 5,000 & 2,000 to 2,500 & 2,400 to 5,000 & No & Steele shale is soft gray marine, Niobrara shale is dark gray and
	contains calcareous zones. \tabularnewline
	\hline 
	Cretaceous & Frontier  & 700 to 900 & 500 to 1,000 & 750 to 1,500 & Yes & Gray sandstone and sandy shale. \tabularnewline
	\hline 
	Cretaceous & Dakota  &  & 300 to 400 &  & Yes & Marine sandstone, tan to buff, fine to medium grained may contain
	carbonaceous shale layer. \tabularnewline
	\hline 
	Jurassic  & Nugget Sandstone  & 400 to 525 & 500 &  & Yes & Grayish to dull red coarse grained cross-bedded quartz sandstone. \tabularnewline
	\hline 
	Triassic  & Chugwater  & 1,275 & 1,500 &  & No & Red shale and siltstone contains gypsum partings near the base. \tabularnewline
	\hline 
	Permian  & Phosphoria  & 275 to 325 & 300 &  & No & Black to dark gray shale, chert and phosphorite. \tabularnewline
	\hline 
	Pennsylvanian  & Tensleep and Amsden and Madison  & 600 to 700 & 750 &  & No & White to gray sandstone containing thin limestone and dolomite partings.
	Red and green shale and dolomite, sandstone near base. \tabularnewline
	\hline 
	Cambrian  & Undifferentiated  & 900 to 1,000 & 1,000 &  & No & Siltstone and quartzite, including Flathead sandstone. \tabularnewline
	\hline
	\hline 
	Precambrian  & Basement  & - & - &  & No & Granites, metamorphic and igneous rocks. \tabularnewline
	\hline
\end{longtable}
\end{landscape}

\begin{landscape}
\begin{longtable}{|c|c|c|c|c|c|c|c|c|c|c|c|}
	\endfirsthead
	\caption{Test of a small longtable on the alternate page.} \\
	\hline
	1 & 2 & 3 & 4 & 5 & 6 & 7 & 8 & 9 & 10 & 11 & 12 \\
	\hline
	A & B & C & D & E & F & G & H & I & J & K & L \\
	\hline
\end{longtable}
\end{landscape}

\subsection{This subsection will follow on a new page that is portrait}
Here it is, just an example

\lipsum[3]




\end{document}
